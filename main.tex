\documentclass[12pt]{article}
\renewcommand{\thesection}{\Roman{section}} 
\renewcommand{\thesubsection}{\thesection.\Roman{subsection}}
%\usepackage[tocindentauto]{tocstyle}
%\usetocstyle{KOMAlike} %the previous line resets it
%\usepackage{natbib}
\usepackage{biblatex}
\addbibresource[]{ref.bib}
\usepackage{url}
\usepackage[utf8]{inputenc}
\usepackage{amsmath}
\usepackage{graphicx}
\usepackage{graphviz}
\usepackage[T1]{fontenc}
\graphicspath{{images/}}
\usepackage{parskip}
\usepackage{fancyhdr}
\usepackage{hyperref}
\usepackage{parskip}
\usepackage{hologo}
\usepackage{listings}
\usepackage{titlesec, blindtext, color}
\usepackage{titling}
\usepackage{tcolorbox}
\usepackage[hmargin=1in,vmargin=1in]{geometry}
\usepackage{float}
\usepackage{tikz}
\usepackage{appendix}
\usepackage{listings} % For code importing
\usepackage{xcolor} % for setting colors
\usepackage{svg}
\usepackage{tocloft}
\renewcommand{\cftsecleader}{\cftdotfill{\cftdotsep}}

\input{arduinoLanguage.tex}

\hypersetup{
	colorlinks=true,
	linkcolor=blue,
	urlcolor=cyan,
}

\lstdefinestyle{customc}{
  belowcaptionskip=1\baselineskip,
  breaklines=true,
  frame=L,
  xleftmargin=\parindent,
  language=C,
  showstringspaces=false,
  basicstyle=\footnotesize\ttfamily,
  keywordstyle=\bfseries\color{green!40!black},
  commentstyle=\itshape\color{purple!40!black},
  identifierstyle=\color{blue},
  stringstyle=\color{orange},
 }

 \lstset{ %
  backgroundcolor=\color{white},   % choose the background color; you must add \usepackage{color} or \usepackage{xcolor}
  basicstyle=\footnotesize,        % the size of the fonts that are used for the code
  breakatwhitespace=false,         % sets if automatic breaks should only happen at whitespace
  breaklines=true,                 % sets automatic line breaking
  captionpos=b,                    % sets the caption-position to bottom
  commentstyle=\color{commentsColor}\textit,    % comment style
  deletekeywords={...},            % if you want to delete keywords from the given language
  escapeinside={\%*}{*)},          % if you want to add LaTeX within your code
  extendedchars=true,              % lets you use non-ASCII characters; for 8-bits encodings only, does not work with UTF-8
  frame=tb,	                   	   % adds a frame around the code
  keepspaces=true,                 % keeps spaces in text, useful for keeping indentation of code (possibly needs columns=flexible)
  keywordstyle=\color{keywordsColor}\bfseries,       % keyword style
  language=Python,                 % the language of the code (can be overrided per snippet)
  otherkeywords={*,...},           % if you want to add more keywords to the set
  numbers=left,                    % where to put the line-numbers; possible values are (none, left, right)
  numbersep=8pt,                   % how far the line-numbers are from the code
  numberstyle=\tiny\color{commentsColor}, % the style that is used for the line-numbers
  rulecolor=\color{black},         % if not set, the frame-color may be changed on line-breaks within not-black text (e.g. comments (green here))
  showspaces=false,                % show spaces everywhere adding particular underscores; it overrides 'showstringspaces'
  showstringspaces=false,          % underline spaces within strings only
  showtabs=false,                  % show tabs within strings adding particular underscores
  stepnumber=1,                    % the step between two line-numbers. If it's 1, each line will be numbered
  stringstyle=\color{stringColor}, % string literal style
  tabsize=2,	                   % sets default tabsize to 2 spaces
  title=\lstname,                  % show the filename of files included with \lstinputlisting; also try caption instead of title
  columns=fixed                    % Using fixed column width (for e.g. nice alignment)
}

\lstdefinestyle{customasm}{
  belowcaptionskip=1\baselineskip,
  frame=L,
  xleftmargin=\parindent,
  language=[x86masm]Assembler,
  basicstyle=\footnotesize\ttfamily,
  commentstyle=\itshape\color{purple!40!black},
}

\lstset{escapechar=@,style=customc}

%\makeatletter
%\let\thetitle\@title

%\let\thedate\@date
%\makeatother

%\pagestyle{fancy}
%\fancyhf{}
%\rhead{\theauthor}
%\lhead{\thetitle}
%\cfoot{\thepage}

\begin{document}
\title{Project Proposal}
%%%%%%%%%%%%%%%%%%%%%%%%%%%%%%%%%%%%%%%%%%%%%%%%%%%%%%%%%%%%%%%%%%%%%%%%%%%%%%%%%%%%%%%%%

\begin{titlepage}
	\centering
    \vspace*{0.5 cm}
    \includegraphics[scale = 0.11]{isu_seal.png}\\[1.0 cm]	% University Logo
    \textsc{\LARGE IOWA STATE UNIVERSITY}\\[2.0 cm]
    \textsc{\large AEROSPACE ENGINEERING DEPARTMENT}\\[0.2 cm]
    \textsc{\large Computational Techniques for Aerospace Design}\\[0.2 cm]
	\textsc{\Large AERE 361}\\[0.5 cm]				% Course Code
	\textsc{\Large Project Proposal}\\[0.2 cm]
	\textsc{\Large TEAM Orion}\\[0.2 cm]
	\rule{\linewidth}{0.2 mm} \\[0.4 cm]
	%{ \huge \bfseries \thetitle}\\
	
	
	\begin{minipage}{0.8\textwidth}
		
			\begin{flushleft} 
			\emph{Team Member Names :} \\
			Bowers, Kayde\linebreak
			Logston, Nathan\linebreak
			Frantz, James\linebreak
			Baer, Judd\linebreak
			Penegor, Riley\linebreak
			
		\end{flushleft}
	\end{minipage}\\[2 cm]
	
	\vfill
	
\end{titlepage}

%%%%%%%%%%%%%%%%%%%%%%%%%%%%%%%%%%%%%%%%%%%%%%%%%%%%%%%%%%%%%%%%%%%%%%%%%%%%%%%%%%%%%%%%%
%\maketitle
\tableofcontents
\pagebreak
%%%%%%%%%%%%%%%%%%%%%%%%%%%%%%%%%%%%%%%%%%%%%%%%%%%%%%%%%%%%%%%%%%%%%%%%%%%%%%%%%%%%%%%%%

\section{ABSTRACT}
%The abstract is a summary of your proposal. In general, your abstract should have enough information so that if I was to copy and paste your abstract into a web site, people would get the general idea of what your proposal is about. It should not go into any heavy detail, just the basics of what your project is about. The who, the what, and the why. You should keep your abstract to 200-400 words. Use this to ``hook in'' your reader.
Lightsabers are iconic weapons from the Star Wars franchise, and this project seeks to bring them to life by creating fully functional and customizable lightsabers. To achieve this, we will be using our knowledge of C and using a Circuit playground express to program various components.  The electronic components will include LED strips, speakers, and a microcontroller board.
The finished product will be a fully functional lightsaber capable of producing various light colors and sound effects. Users can change the color and sound effects using a button or motion sensor. The lightsaber can also be customized by changing the hilt design and programming the microcontroller board.
Overall, this project will showcase our knowledge of C and our ability to problem-solve our design. We want to show that we are capable or creating a working prototype of this iconic weapon. 
\section{INTRODUCTION}
%While the abstract and introduction may seem like it is similar, remember that your abstract should have enough information to stand on its own. The introduction is really the actual start to your proposal. Here you should introduce the project, the people involved and give a short introduction to the why you are doing this. This should be 1-3 paragraphs.
This project aims to create lightsabers using 3D printing and electronic components to produce stunning light and sound effects. The lightsabers will be customizable, and users can create unique designs and program them with various sound and light effects.

The lightsabers will be built with 3D printed components designed to house the electronic components that create the sound and light effects. We will use Arduino boards and programming to control the electronic components and create the desired sound and light effects. The project will also include the development of an app that will enable users to control their lightsaber and change the sound and light effects.

The app will be designed to allow users to adjust the color of the light produced by the lightsaber, change the sound effects, and customize the motion sensors used to activate the lightsaber. The project will be divided into several stages, including the design of the lightsaber components, the programming of the electronic components, the development of the app, and the testing of the finished product.

Overall, this project will demonstrate the power of modern 3D printing and electronic component technology to create stunning lightsabers that can be customized and programmed to produce an array of sound and light effects. It will also showcase the potential of software development to enable users to interact with and control their lightsabers in new and exciting ways.
\section{FEATURES}
%Your Features section must include a listing of at least three key features that makes your project unique. Each item needs to be backed up with a description of what it will do and why. A listing of just three items is not enough, you need to describe what those features are and why your group feels they are needed. For that reason your features should have a paragraph for each key item that describes what that key feature is. A key feature should be something that is significant to your project. For example, a key feature an autopilot system is the ability to be able to set an altitude and the autopilot will automatically set the airspeed. That is a significant feature that has a large impact on that system.

We will start by building the physical lightsaber hilt which will most likely be 3D printed. The hilt will be designed to accommodate the Circuit Playground Express board, which will be the brains of the lightsaber. Alternatively, we might later decide to move most of our electronics onto the user due to space.

Next, we will program the Circuit Playground Express using C to control the sound and light effects. The Circuit PLayground Express is seen in figure \ref{fig:cpx} We will also incorporate sensors such as accelerometers and buttons to enable the user to activate different effects and change the color of the lightsaber.

In addition to creating the lightsabers, we will make our software easy enough to use so that we can change the code to get a larger variety of lightsaber effects. This interface will allow users to adjust settings such as sound volume, the sensitivity of the sensors, and color patterns.


% Below is an example of inserting an image.  Not that LaTex
% will determine the best location for the image.  Make sure
% you replace this image with yours and place a proper caption.
% You can use the \label{name} to name the figure and then reference
% it from your writeup and LaTeX will automatically give it the correct
% number. 
\begin{figure}[!t]
\centering
\includegraphics[width=4.5in]{cpx01.jpg}
\caption{This is the Circuit Playground Express}
\label{fig:cpx}
\end{figure}

\section{PROBLEM STATEMENT}

Custom lightsabers are often sought by kids and star wars enthusiasts alike. However, there are currently limitations on what one lightsaber can do. These limitations include only being able to be one color, having fixed sounds, and either being stuck to one extend and retract speed or not being able to extend and retract at all. These limitations often force consumers to purchase multiple lightsabers or parts to switch out while only being able to use one option at a time. These quickly increase the price of custom lightsabers, causing some consumers to forfeit certain options. 

%Here you will go into more detail on what problem you hope to solve or address.  You should discuss what the problem is and why it is important to solve it. In this section, you need to be clear on what the problem is, so do not think of this as a ``light'' section. It helps to define your project.

%Your team needs to do some research into the problem at hand. Because of that, you should have around two to three references that you are pulling from. There are lots of places you can find references from including the ISU library and Google Scholar. I would also suggest looking at Adafruit's website, as you may find inspiration or looking to improve something already there. Remember to cite your sources though. If you find something online, that can often be citation.

%When you create your ``ref.bib'' file, don't forget to follow the standards for a BiBTex file. Certain things like webistes requires certain keywords for it to render properly. There are lots of sources online to help with this and many places like the ISU Library and Google Scholar can also generate text that is compatible with a BiBTex file. Once you have your Bib file ready, don't forget to cite your citations in your proposal like this \cite{einstein} or this \cite{dirac}.

\section{PROBLEM SOLUTION}

Our solution involves using the LIS3DH accelerometer that is included within the Circuit Playground. This sensor will give us acceleration data that we can use to detect user swing and thrust inputs. The Circuit Playground can read .wav files, which we can use to play idle, swing, and clash sounds through the Circuit Playground's internal speaker. This speaker may not be loud enough, so an external speaker and driver may be required. 

The lightsaber can be ignited with a single switch on the side of the assembly, with a potentiometer beneath it that can control the extension and retraction rate of the saber. Many Star Wars fans have different preferences for older lightsaber styles that have slow extensions, and a different flicker pattern than the new ones. We would like our solution to fill any personal preferences the end user would have. 

A secondary button would allow the user to access various features or extra sound effects, such as blaster deflections or opponent lock sounds. The button could also be used to access a sound-based menu that the user can use to change parameters, such as the sound font or volume. In this menu, the ignition button can be used to cycle the parameter. 

\begin{figure}[!t]
  \centering
  \includegraphics[width=4.5in]{Circuit_Diagram.png}
  \caption{Propsed circuit diagram with 2 buttons, potentiometer, and neopixel strip.}
  \label{fig:CircuitDiagram}
  \end{figure}

%You must also include a table that lists all the parts that you wish to have. As announced in class, you will have the parts listed in Table \ref{table:parts_list}. We have plenty of two additional parts. Those are a conductive adhesive strip and a neopixel strip. I do have some other parts, such as arcade buttons and some additional sensors. You can certainly ask for something, and I will see what I can do. Change the table below to reflect the parts you are requesting.

\begin{table}[ht]
  \caption{Parts List}
  \label{table:parts_list}
  \begin{center}
  \begin{tabular}{|p{3in}|c|}
  
  \hline
  Part description & Qty\\
  \hline
  \hline
  Adafruit Circuit Playground Express & 1 \\
  \hline
  AAA Battery Holder & 1 \\
  \hline
  USB Cable & 1 \\
  \hline
  Button & 2 \\
  \hline
  Speaker & 1 \\
  \hline
  Potentiometer & 1 \\
  \hline
  Two Position switch & 1 \\
  \hline
  Neopixel Strip & 1 meter \\
  \hline
  \end{tabular}
  \end{center}
  \end{table}

There are several different ways the problem can be solved using code, but there are several resources online that provide examples that we can use to start coding for this project. On Adafruit's website is an article titled \emph{CircuitPython Made Easy on Circuit Playground Express and Bluefruit}, which provides an example of how to play sound files. This example uses the cp function from the adafruit\_circuitplayground library. \cite{adafruitPlayfile} 
  %Finally, you can also include any pseudo code or any code snippets you have gathered so far.  This is not required, but if you found some starter code or came up with some ideas for the code, put it here. If you want to embed code into \LaTeX, you can use the example below on how to do this in \LaTeX.

\begin{lstlisting}[language=Arduino]
  # SPDX-FileCopyrightText: 2021 ladyada for Adafruit Industries
  # SPDX-License-Identifier: MIT
  
  """THIS EXAMPLE REQUIRES A WAV FILE FROM THE examples FOLDER IN THE
  Adafruit_CircuitPython_CircuitPlayground REPO found at:
  https://github.com/adafruit/Adafruit_CircuitPython_CircuitPlayground/tree/main/examples
  
  Copy the "dip.wav" and "rise.wav" files to your CIRCUITPY drive.
  
  Once the files are copied, this example plays a different wav file for each button pressed!"""
  from adafruit_circuitplayground import cp
  
  while True:
      if cp.button_a:
          cp.play_file("dip.wav")
      if cp.button_b:
          cp.play_file("rise.wav")
\end{lstlisting}

Another good resource comes again from adafruit's website, and is part of the circuit playground's tutorial wiki. It includes some functions we can use to read data from the accelerometer, such as \lstinline{CircuitPlayground.motionX()} and setting the accelerometer's sensitivity with \lstinline{CircuitPlayground.setAccelRange(range)}. \cite{adafruitAccelerometer}

\section{CONCLUSION}
%Finally, wrap up your proposal. This only needs to be one or two paragraphs, but it should conclude with what you plan to do and the why and how. Yes, this may seem repetitive, but that is intentional. Do not forget to update your references as those will appear below in a seperate page.
This project proposal has presented a unique and exciting opportunity to use the programming language C and the Circuit Playground Express to create programmable lightsabers.The programming of the Circuit Playground Express will enable us to control the lightsaber's color and sound effects and incorporate various sensors to enable user interaction. Additionally, the project will create a software interface that will allow users to customize and program their own lightsaber effects.

This project will not only showcase the capabilities of C and modern technology in creating a visually appealing lightsaber, but it will also provide a unique opportunity for learning and creativity. By using C programming and the Circuit Playground Express, we will be able to mess with microcontrollers and explore the possibilities of integrating different sensors and software to create a unique and customized lightsaber experience.
\newpage
\section{References}
\printbibliography[heading=subbibintoc]
%\bibliographystyle{plain}
%\bibliography{ref}

\end{document}
